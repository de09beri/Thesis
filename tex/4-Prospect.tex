\chapter{PROSPECT}

The scientific community's understanding of neutrinos has come a long way from Pauli's initial proposition of its existence in 1930. 
Though three active neutrino flavors and their behaviors are understood and well included in the Standard Model of particle physics, recent anomalies in reactor neutrino experiment results hint at the possibility of new physics. 
The discovery of an eV-scale sterile neutrino would have wide ranging impacts on the field of neutrino physics and future experiments.

The Precision Reactor Oscillation and Spectrum Experiment (PROSPECT) is designed to address the reactor antineutrino anomaly by performing a reactor-model independent search for short-baseline $\bar{\nu_{e}}$ oscillations and making a high precise measurement of the $^{235}$U $\bar{\nu_{e}}$ energy spectrum at a highly-enriched uranium (HEU) research reactor \cite{LongNIM}.
Located at the High Flux Isotope Reactor (HFIR) at Oak Ridge National Laboratory (ORNL) in Tennessee PROSPECT also demonstrates successful application of techniques for antineutrino detection at the surface with little overburden. 
PROSPECT collected data from May to December of 2018 and the first oscillation and spectrum results, with 33 and 40.3 live-days of reactor on time respectively, can be found in Ref.\cite{PhysRevLett.121.251802,Ashenfelter:2018jrx}.

\section{Experimental Site}
\subsection{HFIR}

HFIR is a compact research reactor that burns highly enriched uranium fuel ($^{235}$U), meaning that $>$ 99\% of fissions during a reactor cycle will be from $^{235}$U.
The HFIR core consists of two concentric fuel elements with an outer diameter of 0.435 m and a height of 0.508 m, surrounded by control elements and Beryllium reflectors as shown in Figure~\ref{fig:hfir}. 
The reactor typically operates at 85 MW for seven 24-day cycles per year for a duty cycle of $\sim$46\%.

\begin{figure}[t]
	\centering
	\includegraphics[width=0.9\linewidth]{tex/4-prospect-images/HFIR}
	\caption[The HFIR core and flux distribution]{(a) The HFIR core, showing the inner and outer fuel elements and the flux trap region, as well as the control elements and Beryllium reflectors. (b) The relative fission density distribution at the start of a cycle. \cite{HFIRTech}}
	\label{fig:hfir}
\end{figure}

\subsection{Backgrounds at HFIR}

The PROSPECT detector is located at ground level $\sim$7 m from the HFIR core and separated from the reactor water pool by a 1 m thick concrete wall, see Figure~\ref{fig:shielding}.
The proximity to the reactor and lack of overburden introduces a significant level of background events from the reactor and cosmogenic sources. 
These events can be classified into two categories, singles that are mainly due to gammas and coincident events from neutron recoil and captures. 
Extensive studies on the types and rate of background events at the detector site can be found in Refs.\cite{Ashenfelter:2015tpm,Heffron,Hackett}.

The largest source of gamma backgrounds was discovered to originate in the reactor pool wall, specifically an unused beam line that lies directly in front of the detector. 
In order to lower these backgrounds a lead shield wall (3.0 m wide, 2.1 m tall, and on average 0.10 m thick), along with shorter flanking walls on each side and a mini-wall placed at the opening of the beam line, was installed between the pool wall and the detector.
Other background events were shown to come from neutron beam-lines and scattering experiments existing below the detector site, but most of these are suppressed by a concrete monolith that the detector sits on.

The thermal neutron rate was measured to be $\sim$2/cm$^2$/s during reactor operation \cite{Ashenfelter:2015tpm}, so layers of shielding containing $^{10}$B, which has a large thermal neutron cross-section and minimal gamma emission, were used in the passive shielding that surrounds the detector (see Section\todo{Add section cite}). 
The locally installed shield wall, along with the addition of passive shielding, resulted in a sufficient suppression of background such that a better than one-to-one signal to background ratio was achieved. 

\begin{figure}[t]
	\centering
	\includegraphics[width=0.5\linewidth]{tex/4-prospect-images/Shielding}
	\caption[Layout of PROSPECT at HFIR]{Layout of the PROSPECT experiment. The detector is installed in the HFIR Experiment Room next to the water pool and 5 m above the HFIR reactor core (red). The floor below contains multiple neutron beam-lines and scattering experiments.}
	\label{fig:shielding}
\end{figure}

\section{Design}

The PROSPECT antineutrino detector (AD) consists of a segmented inner detector filled with $^6$Li doped liquid scintillator (LiLS), contained in an acrylic and aluminum tank, and surrounded by layers of passive shielding. 
The active detector is made up of a 14  $\times$ 11 array of optically separated segments, viewed on each end by a photomultiplier tube (PMT) enclosed in an acrylic housing. 
The AD is placed with the segments parallel to the reactor pool wall, $\sim$7 m from the reactor core, and measures $\sim$3 m tall including all shielding \todo{Check this measurement}. 
See Figure~\ref{fig:ad} for a schematic of the detector. 

\begin{figure}[t]
	\centering
	\includegraphics[width=0.8\linewidth]{tex/4-prospect-images/AD}
	\caption[Schematic of the PROSPECT detector]{A cutaway view of the PROSPECT detector, including the inner detector, outer containment vessels, and passive shielding.}
	\label{fig:ad}
\end{figure}

%===========================================================
\subsection{Inner Detector}

\subsubsection{PMT Housings}

A total of 308 PMTs are installed in the AD; 240 Hamamatsu R6594 SEL PMTs used in the inner segments (fiducial volume) and 68 ADIT Electron Tubes 9372KB PMTs used in the outer segments as shown in Figure~\ref{fig:adcrosssection}.
Each PMT is mounted inside a rectangular acrylic housing facing a clear 144-mm-square front window constructed from ultraviolet transmitting (UVT) acrylic, allowing them to exist inside of the LiLS. 
Conical reflectors were installed at the face of the housing to improve light collection efficiency in the corners. 
The housing is filled with optical grade mineral oil and sealed with an O-ring and a 32-mm-thick back plug. 
For a detailed drawing of the PMT housing module see Figure~\ref{fig:pmthousing}.
For more information on the PMT housing design and construction see Ref.\cite{LongNIM}.

\begin{figure}[h]
	\centering
	\begin{minipage}[t]{0.48\linewidth}
		\centering
		\includegraphics[width=0.95\linewidth]{tex/4-prospect-images/ADCrossSection}
		\caption[Cross-section of inner detector]{A cross-section of the inner AD showing 68 ET PMTs (red) in the outer columns and top row and 240 Hamamatsu PMTs (blue) in the remaining segments.}
		\label{fig:adcrosssection}
	\end{minipage}
	\begin{minipage}[t]{0.48\linewidth}
		\centering
		\includegraphics[width=0.95\linewidth]{tex/4-prospect-images/PMTHousing}
		\caption[PMT housing module]{A PMT housing module.}
		\label{fig:pmthousing}
	\end{minipage}
\end{figure}


\subsubsection{Optical Grid}

The active volume of the detector, measuring 2.045 m wide $\times$ 1.607 m high $\times$ 1.176 m long, is separated into 154 optically separated long segments with a 0.145 m $\times$ 0.145 m square cross-sectional area. 
The optical grid that creates the individual segments consists of low-mass, highly specularly reflective optical separators held in position by white 3D-printed support rods. 
The optical separators (reflectors) are composed of a carbon fiber backbone covered on both sides with adhesive-backed 3M DF2000MA specularly reflecting film, an optically clear adhesive film, and a thin surface layer of fluorinated ethylene propylene (FEP) film.
Two types of pinwheel shaped support rods were produced and strung on acrylic rods to grip the reflectors and hold them in place and separate the PMT housings from each other.
The pinwheels were 3D printed using white-dyed 100-micron polylactic acid (PLA) filament and are pictured in Figure~\ref{fig:pinwheels}.
For more details on the fabrication of the optical grid see Ref.\cite{Ashenfelter:2019lbf}.

Each segment contains a PMT housing at each end and four reflectors held in place by pinwheel rods that extend from one PMT to the other, as show in Figure~\ref{fig:fullsegment}.
The front windows of the PMT housings protrude $\sim$ 1 cm into the optical grid, minimizing cross-talk 
between segments. 
Figure~\ref{fig:rowassembly} shows the assembly of the top row of the detector, demonstrating the placement of the housings and optical grid. 


\begin{figure}[]
	\centering
	\begin{minipage}[t]{0.4\linewidth}
		\centering
		\includegraphics[width=0.8\linewidth]{tex/4-prospect-images/Pinwheels}
		\caption[Pinwheels]{Representative pinwheel types. (a) Central pinwheel - Three tabs per side hold the optical
		separator in place. (b) End pinwheel - spacer arms separate the PMT housing bodies and support the
		pinwheel string.}
		\label{fig:pinwheels}
	\end{minipage}
	\begin{minipage}[t]{0.48\linewidth}
		\centering
		\includegraphics[width=0.98\linewidth]{tex/4-prospect-images/FullSegment}
		\caption{Three complete segments, including PMT housings at each end with reflectors kept in in place between segments by pinwheel rods.}
		\label{fig:fullsegment}
	\end{minipage}
\end{figure}

\begin{figure}[]
	\centering
	\includegraphics[width=0.7\linewidth]{tex/4-prospect-images/RowAssembly}
	\caption[Construction of a row]{Assembly of the top row of the PROSPECT AD, demonstrating the placement of the PMT housings and optical grid.}
	\label{fig:rowassembly}
\end{figure}


\subsubsection{Segment Supports}

While the optical grid creates the inner volume segmentation, acrylic segment supports hold the total volume in place and determine the size of the active volume.
A slab of ship-lack style acrylic underneath the bottom row of segments position them at a 5.5$^{\circ}$ tilt with a 0.146 m pitch.
Horizontal planks are screwed onto the backs of the PMT housings and attach together along the sides of the volume, while side walls constrain the outer pinwheel rows. 
Baffles at the top of the detector tie the four surrounding walls together and keep the top reflector layer in place. 



\begin{figure}[h]
	\centering
	\includegraphics[width=0.7\linewidth]{tex/4-prospect-images/CompletedDetector}
	\caption{}
	\label{fig:completeddetector}
\end{figure}


\subsubsection{Liquid Scintillator}



\subsubsection{Calibration}


%===========================================================
\subsection{Containment Vessels and Shielding}




%===========================================================
%===========================================================
\section{Detecting Inverse Beta Decays}

\section{From Signal to Result}


