\begin{abstract} 
\thispagestyle{plain}
\addcontentsline{toc}{chapter}{ABSTRACT}
\doublespacing

The Precision Reactor Oscillation and SPECTrum Experiment (PROSPECT) is designed to probe short baseline oscillations of electron antineutrinos in search of $\textrm{eV-scale}$ sterile neutrinos and precisely measure the \textsuperscript{235}U reactor antineutrino spectrum from the High Flux Isotope Reactor (HFIR) at Oak Ridge National Laboratory. 
The PROSPECT antineutrino detector (AD) provides excellent background rejection and position resolution due to its segmented design and use of \textsuperscript{6}Li-loaded liquid scintillator. 
In order to understand relative volume variation effects, which could affect an oscillation measurement, \textsuperscript{227}Ac was added as a calibration source that was dissolved isotropically throughout the liquid scintillator. 
Using the correlated production of alphas from \textsuperscript{219}Rn $\rightarrow$ \textsuperscript{215}Po $\rightarrow$  \textsuperscript{211}Pb in the \textsuperscript{227}Ac decay chain I measured the rate of \textsuperscript{227}Ac in each segment of the detector as well as the decay rate of \Ac events over the lifetime of the detector.
The measured \Ac half-life suggests a rate of events falling 1.56 $\pm$ 0.21\% faster than expectation. 
The results of these studies were then applied as corrections to the measurement of antineutrino event rates as a function of distance from the reactor.
This thesis will present the testing of \Ac as a calibration source before its addition to the AD, analysis methods, results of \Ac in the AD, and its application to the oscillation analysis. 

\end{abstract}
