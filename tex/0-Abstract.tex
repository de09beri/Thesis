\begin{abstract} 
\thispagestyle{plain}
\addcontentsline{toc}{chapter}{ABSTRACT}
\doublespacing

The Precision Reactor Oscillation and SPECTrum Experiment (PROSPECT) is designed to probe short baseline oscillations of antineutrinos in search of eV-scale sterile neutrinos and precisely measure the \textsuperscript{235}U reactor antineutrino spectrum from the High Flux Isotope Reactor (HFIR) at Oak Ridge National Laboratory. 
The PROSPECT antineutrino detector (AD) provides excellent background rejection and position resolution due to its segmented design and use of \textsuperscript{6}Li-loaded liquid scintillator. 
Due to characteristics of its decay chain, \textsuperscript{227}Ac was added as a calibration source that was dissolved isotropically throughout the liquid scintillator. 
Using the correlated production of alphas from \textsuperscript{219}Rn $\rightarrow$ \textsuperscript{215}Po $\rightarrow$  \textsuperscript{211}Pb in the \textsuperscript{227}Ac decay chain we can measure the rate of \textsuperscript{227}Ac in each segment of the detector. 
This allows us to precisely determine the relative segment to segment volume variation to 1\%. 
These measurements can then be applied as corrections to measurements of neutrino oscillation through the PROSPECT AD.

\end{abstract}
