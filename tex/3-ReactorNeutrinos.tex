\chapter{\uppercase{Reactor Neutrinos}}

Nuclear reactors are a pure source of electron antineutrinos, $\bar{\nu_e}$, as a result of the fission of isotopes used in the reactor fuel. The first neutrino was discovered using the nuclear reactor at the Savannah River Plant, and reactor sites continue to be popular homes for neutrino detectors. 
In order to perform precision reactor neutrino studies it is important to understand the reactor neutrino flux and spectrum.

\section{Production of Reactor Neutrinos}

Nuclear reactors are powered by the fission of uranium and plutonium isotopes in their cores. 
Specifically, in a power reactor, 99.9\% of the power comes from the fission of $^{235}$U, $^{239}$Pu, $^{241}$Pu, and $^{238}$U isotopes. 
The chain reaction begins with a neutron colliding with a nucleus of one of the isotopes. 
This causes the nucleus to split into two fragments, usually of unequal mass, creating an unstable system.
In order to reach stability neutrons have to transform into protons, a process only accomplished through $\beta$ decay, see Figure~\ref{fig:nucchart}.
Each beta decay produces an electron and corresponding electron antineutrino. 
In general a nuclear reactor will produce $\sim 6 \times 10^{20} \bar{\nu_e}$ per GW of thermal energy power \cite{HayesVogel}.

\begin{figure}[h]
	\centering
	\includegraphics[width=0.7\linewidth]{tex/3-reactorneutrinos-images/NuclideChart_U235}
	\caption{A schematic of the fission of $^{235}$U \cite{NucChart}. After collision with a neutron $^{235}$U will split into two unstable nuclei (pink arrows) which will then $\beta$ decay (white arrows) until stable.}
	\label{fig:nucchart}
\end{figure}


\section{Measuring the Reactor Antineutrino Flux and Spectrum}

The total $\bar{\nu_{e}}$ flux, $S(E_\nu)$, produced by a nuclear reactor can be expressed as the sum over the spectra of the dominant fissioning isotopes,

\begin{equation}
	S(E_\nu) = \frac{W_{th}}{\sum_{i}(f_i/F)e_i}\sum_{i}\frac{f_i}{F}\left(\frac{dN_i}{dE_\nu}\right) ,
\end{equation}

where $f_i/F$ is the fission fraction for each given isotope $i$, $W_{th}$ is the reactor thermal energy, $e_i$ is the 
average energy released per fission by each isotope, and $dN_i/dE_\nu$ is the cumulative $\bar{\nu_e}$ spectrum of $i$ normalized per fission.

There are two methods to determine the $\bar{\nu_e}$ spectrum, \textit{ab initio} summation and electron spectrum conversion.
In the \textit{ab initio} approach the spectrum is determined by summing the contributions of all $\beta$-decay branches of all fission fragments,

\begin{equation}
	\frac{dN_i}{dE_{\bar{\nu}}} =  \sum_{n}Y_n(Z,A,t)\sum_{n,i}b_{n,i}(E^i_0)P_{\bar{\nu}}(E_{\bar{\nu}},E^i_0,Z) ,
\end{equation}

where $Y_n(Z,A,t)$ is the number of $\beta$ decays of the fission fragment $Z, A$ at time $t$, $b_{n,i}(E^i_0)$ are the branching ratios with endpoint energies $E^i_0$, and $P_{\bar{\nu}}(E_{\bar{\nu}},E^i_0,Z)$ is the normalized $\bar{\nu_e}$ spectrum for the branch $n, i$.
This method relies on nuclear databases, such as the Evaluated Nuclear Data File (ENDF) and Joint Evaluated Fission and Fusion (JEFF) databases, for information about the branching ratios and decay energies. 
The antineutrino spectrum for the four main reactor isotopes calculated using \textit{ab initio} summation was done in Ref.~\cite{HayesVogel} and the result can be seen in Figure~\ref{fig:spectrum}. 
Though seemingly straightforward, this approach comes with some caveats.
The shear number of daughter isotopes ($>$1000) and individual $\beta$ decay branches ($>$6000) make the summation non-trivial.
This, along with the fact that not all branching ratios are known, and that the fission yields have been determined by several different database groups but don't always agree and have large uncertainties bring into question the validity of using only this method. 

\begin{figure}[t]
	\centering
	\includegraphics[width=0.7\linewidth]{tex/3-reactorneutrinos-images/Spectrum}
	\caption{The $\bar{\nu_e}$ spectrum predicted by the summation method using the JEFF-3.1.1 database fission fragment yields and the ENDF/B-VII.1 decay library \cite{HayesVogel}.}
	\label{fig:spectrum}
\end{figure}

The other approach to determine the $\bar{\nu_{e}}$ spectrum, the conversion method, relies on converting a measured electron spectrum into an antineutrino spectrum. 
This involves fitting an experimentally defined total beta spectrum with individual beta spectrum  according to their amplitudes, $a_i$, 

\begin{equation}
	\frac{dN_i}{dE_e} = \sum_{i}a_iP(E,E^i_0,Z)
\end{equation}

The conversion to the antineutrino spectrum is then accomplished by replacing the energy $E_e$ in each branch by $E_0 - E_{\bar{\nu}}$, because the electron and the $\bar{\nu_e}$ share the total energy of each $\beta$-decay branch.
The flux per fission is then given as the sum of $\bar{\nu_e}$ spectrum converted from each virtual $\beta$ branch,

\begin{equation}
	\frac{dN_i}{dE_{\bar{\nu}}} = \sum_{i}a_iP(E^i_0-E,E^i_0)
\end{equation}

\todo{New figure?}
\begin{figure}
	\centering
	\includegraphics[width=0.7\linewidth]{tex/3-reactorneutrinos-images/betaspecconversion_fixed}
	\caption{}
	\label{fig:betaspecconversionfixed}
\end{figure}

The electron spectra for $^{235}$U, $^{239}$Pu, and $^{241}$Pu were measured at the Institut
Laue-Langevin (ILL) reactor in Grenoble, France in the 1980s \cite{VonFeilitzsch:1982jw,Schreckenbach:1985ep,Hahn:1989zr}, while the spectrum of $^{238}$U was more recently measured at the neutron source FRMII in Garching, Germany \cite{Haag:2013raa}.



\section{Detection of Reactor Neutrinos}

Though there are several methods that can be used to detect reactor neutrinos, including charge-current ($\bar{\nu_e} + d \rightarrow n + n + e^+$), neutral-current ($\bar{\nu_e} + d \rightarrow n + p + \bar{\nu_e}$), and antineutrino-electron elastic scattering ($\bar{\nu_e} + e^- \rightarrow \bar{\nu_e} + e^-$), the one employed by most experiments is IBD ($\bar{\nu_e} + p \rightarrow e^+ + n$).
The IBD reaction energy threshold is 1.8 MeV and the cross section is relatively high, $\sim 63 \times 10^{-44} \textrm{cm}^2/\textrm{fisson}$ integrated over the entire reactor neutrino energy spectrum \cite{Qian:2018wid}, and can be written as

\begin{equation}
	\sigma^{(0)} \simeq 9.52 \times \left(\frac{E_e^{(0)}p_e^{(0)}}{\textrm{MeV}^2}\right) \times 10^{-44}\textrm{cm}^2
\end{equation}

where $E_e$ and $p_e$ are the energy and momentum of the final-state positron. 

An IBD event is selected by a pair of coincident signals consisting of a positron ionization and annihilation as the prompt signal and a time delayed neutron capture on a proton or nucleus as the delay signal. 
The neutron energy can be backtracked from the prompt signal as
\begin{equation}	
	E_{\bar{\nu}} = E_{prompt} + 0.78~\textrm{MeV} + T_n
\end{equation}
where $T_n$ is the kinetic energy of the recoil neutron which is much smaller than the energy of the neutrino and can therefore be ignored in most cases. 
The IBD cross-section increases with energy, whereas the $\bar{\nu_{e}}$ spectrum decreases with energy creating a detected energy spectrum that peaks around 3.8 MeV and dies off after $\sim$8 MeV, as seen in Figure~\ref{fig:vogel-fig02}. 

In addition to great background rejection and good reconstruction of the neutrino energy, the IBD method of detecting neutrinos also allows the use of liquid scintillators and water as detection mediums. 

\begin{figure}[h]
	\centering
	\includegraphics[width=0.7\linewidth]{tex/3-reactorneutrinos-images/vogel-fig02}
	\caption{The IBD spectrum (curve (a)) measured by a 12-ton fiducial mass detector located 0.8 km from a 12-GW$_{th}$ power reactor along with the reactor flux (curve (b)) and IBD cross section (curve (c)) as a function of energy \cite{PDG}.}
	\label{fig:vogel-fig02}
\end{figure}




\section{The Reactor Antineutrino Anomaly}






