\chapter{CONCLUSION}

The Precision Reactor and Spectrum Experiment (PROSPECT) was designed to probe short baseline oscillations of electron antineutrinos from the High Flux Isotope Reactor in search of an eV-scale sterile neutrino.
Making use a segmented design PROSPECT was able to make a reactor model independent measurement of prompt energy spectra across baselines.
With 33 days of reactor-on data PROSPECT was able to disfavor the best-fit point of the reactor antineutrino anomaly at 2$\sigma$ confidence level. 
PROSPECT has also been able to make the, currently, most precise measurement of the $^{235}$U antineutrino spectrum measurement. 

In order to measure relative segment-to-segment volume variations \Ac was added to the liquid scintillator to provide a source of $\alpha$ decays.
By measuring the rate of decay I was able to make relative volume measurements to around 1\%.
As well as providing a proxy for volume, \Ac was also proven useful for tracking the position and energy resolution of the detector. 
With a large statistics data set, in which understanding the relative volumes could become an important systematic, I have proven that added a radioactive source is a viable method for measuring these volume with accuracy. 

