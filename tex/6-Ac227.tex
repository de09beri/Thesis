\chapter{\uppercase{$^{227}$Ac as a Calibration Source}} \label{ch:Ac}

\section{Motivation}

In the absence of an eV-scale sterile neutrino PROSPECT should measure IBD rates that fall like one over distance from the reactor squared. 
If sterile neutrino oscillation was detected, after one year PROSPECT would measure the spectrum seen in Figure~\ref{fig:lovere1yr}, given a mass splitting of 1.78 eV$^2$, .
However, a situation could occur in which a sterile neutrino did not exist, or could not be measured with PROSPECT, and an oscillation is still measured.
This could happen if there are segment to segment volume variations throughout the detector, mimicking an oscillation signal.
Therefore, it becomes important that the product of efficiency$\times$volume for all segments is well known. 

\begin{figure}[h]
	\centering
	\includegraphics[width=0.7\linewidth]{tex/6-ac227-images/LoverE_1yr}
	\caption[Monte Carlo neutrino spectrum as a function of L/E]{The ratio of the oscillated to un-oscillated neutrino spectrum as a function of L/E that would be observed by PROSPECT after 1 year if a sterile neutrino signal was detected \cite{PSurukuchi:1534}.}
	\label{fig:lovere1yr}
\end{figure}


This measurement can be accomplished if an event source is uniformly distributed throughout the active volume of the detector. 
By measuring the rate of this source in each segment the relative volumes can be compared and tracked through time.
\Ac was chosen as the source and a chloride solution was prepared and dissolved in the liquid scintillator to ensure uniform distribution.

\Ac was chosen for several reasons. First, because an \AaAa coincidence occurs in its decay chain, specifically $^{219}$Rn $\rightarrow ^{215}$Po + $\alpha \rightarrow ^{211}$Pb + \Aa, as highlighted in Figure~\ref{fig:ac227chain}.
\Rn has a half-life of 3.96 $\pm$ 0.01 s and $\alpha$-decays 100\% of the time, while \Po has a half-life of 1.781 $\pm$ 0.005 ms and $\alpha$-decays  99.99977(2)\% of the time \cite{ENSDF}, so the \AaAa coincidences happen frequently.
The \Aa decay of \Po is mono-energetic at 7.39 MeV which results in a $\sim$0.78 MeVee signal after quenching, well removed from nLi captures that occur around 0.5 MeVee. 
In addition, there are no corresponding gammas with the \Po decay, making this a very clean and well defined signal.
The \Rn \Aa decays are not so nice, as there are four dominant alpha energies and three dominant gamma energies for these decays, as listed in Table~\ref{tab:RnPoE}, but the use of time, energy, and PSD cuts make them easy to pair with corresponding \Po decays.

\begin{figure}[hb]
	\centering
	\includegraphics[width=0.6\linewidth]{tex/6-ac227-images/Ac227Chain}
	\caption[\Ac decay chain]{The full decay chain of \Ac (a daughter of $^{235}$U), in which the \AaAa coincidence of interest is highlighted \cite{Kirby2011}.}
	\label{fig:ac227chain}
\end{figure}

\begin{table}[h]
	\centering
\begin{tabular}{|c|c|c|c|c|c|}
	\hline 
	& E$_\alpha$ [keV] & I$_\alpha$ \% &  & E$_\gamma$ [keV] & I$_\gamma$ \% \\ 
	\hline 
	\Rn & 6425.0(10) & 7.5(6) &  & 271.23(1) & 10.8(6) \\ 
	%\hline 
	& 6530(2) & 0.110(10) &  & 401.81(1) & 6.6(4) \\ 
	%\hline 
	& 6552.6(10) & 12.9(6) &  & 130.60(3) & 0.13(9) \\ 
	%\hline 
	& 6819.1(3) & 79.4(10) &  &  &  \\ 
	\hline 
	\Po & 7386.1(8) & 99.999770(20) &  &  &  \\ 
	\hline 
\end{tabular} 
\caption[$\alpha$ and $\gamma$ energies of \Rn and \Po decays]{Energy and absolute intensity of dominant $\alpha$ and $\gamma$ decay radiation for \Rn and \Po. Decay energies not listed here have an intensity of $< $0.05\%.}
\label{tab:RnPoE}
\end{table}

%\begin{figure}[h]
%	\centering
%	\includegraphics[width=0.95\linewidth]{tex/6-ac227-images/Actinium_agSpectrum}
%	\caption[]{\cite{Kirby2011}}
%	\label{fig:actiniumagspectrum}
%\end{figure}


\section{Material Compatibility}

Before \Ac could be added to the PROSPECT detector, it had to be determined that \Ac and its daughters would not adsorb onto detector materials.
If it was adsorbed then it would not be a uniform source in the detector, nullifying its use as a method to track relative efficiency$\times$volume throughout the detector.
To test this six material samples were placed in vials of $^6$Li-LS spiked with \Ac. The rate of \Ac in each sample vial and one reference vial with no material was measured and tracked over a period of 6 months. 

\subsection{Material and Scintillator Preparation}
The materials tested were: ultraviolet transmitting (UVT) acrylic, flourinated ethylene propylene (FEP), polylactide (PLA), polyether ether ketone (PEEK), a RG 188 cable, and viton o-rings. See Table~\ref{tab:materials} for a list of their uses in the detector and sample sizes.
To prepare the materials they were all placed in a single beaker with ultra-pure water and cleaned ultrasonically for 30 minutes.
They were then transferred to a watch glass and placed in a 50 C over for two hours.
After drying they were placed in empty 12 mL vials.

The \Ac used to spike the scintillator was obtained from Eckert and Ziegler as a solution of 3.711$\times$10$^4$ Bq $\pm$ 1.32\% of \Ac in 10.22710 g of 1 M HCl, measured on September 6, 2016.
0.503 g of this solution was added to 192 g of $^6$Li-LS on December 15, 2016.
With a half-life of 21.772 $\pm$ 0.003 yrs \cite{ENSDF}, the activity of the \Ac solution before adding to the LiLS was 36788 Bq, yielding a final activity of 94.2 Bq/10 g. 
This is the stock solution from which all LS will be taken for the material studies and later on for spiking the detector.


\begin{table}[H]
	\centering
\begin{tabular}{|p{0.17\textwidth}|p{0.45\textwidth}|p{0.33\textwidth}|}
	\hline 
	\textbf{Material} & \textbf{Detector Use} & \textbf{Sample Size} \\ 
	\hline 
	UVT Acrylic & Front window of PMT housing & 1.0 $\times$ 1.15 $\times$ 0.1 cm$^3$ \\ 
	\hline 
	FEP  & Film on optical separators & 1.5 $\times$ 1.5 cm$^2$, 3 mm thick \\ 
	\hline 
	PLA & 3D printed pinwheels & 10 disks; 0.5 cm diameter, 0.1 cm thick \\ 
	\hline 
	PEEK & Seal plugs through which the high voltage and signal cables were threaded. Screws used to bolt together segment supports. Spacers at the base of the acrylic tank. & 1 Nut; ID 0.5 cm, small OD 1cm, large OD 1.1cm, thickness 0.5 cm \\ 
	\hline 
	RG188 Cable & High voltage and signal cables & 4.5'' long \\ 
	\hline 
	Viton O-ring & Seal back plugs of PMT housings and seal acrylic tank & 10 O-rings; OD 6mm, ID ~3mm, thickness 1.5mm \\ 
	\hline 
\end{tabular} 
\caption{Samples used to test if \Ac or its daughters would adsorb onto detector materials.}
\label{tab:materials}
\end{table}

\begin{tabular}{|c|c|c|c|}
	\hline 
	Material & Date Filled & Weight of LS (g) & Expected Activity on Fill Day (Bq) \\ 
	\hline 
	Reference & 12/15/2016 & 10.030 & 94.5 \\ 
	\hline 
	UVT Acrylic & 02/24/2017 & 9.98 & 93.4 \\ 
	\hline 
	FEP &  & 9.98 &  \\ 
	\hline 
	PLA &  & 9.999 &  \\ 
	\hline 
	PEEK &  & 9.99 &  \\ 
	\hline 
	RG188 Cable &  & 9.981 &  \\ 
	\hline 
	Viton O-ring &  & 10.011 &  \\ 
	\hline 
\end{tabular} 


\begin{figure}[h]
	\centering
	\includegraphics[width=0.7\linewidth]{tex/6-ac227-images/BNL/Samples}
	\caption{}
	\label{fig:samples}
\end{figure}

\subsection{Measurement Scheme}

\begin{figure}[H]
	\centering
	\includegraphics[width=0.7\linewidth]{tex/6-ac227-images/BNL/BlackBox}
	\caption{}
	\label{fig:blackbox}
\end{figure}


\begin{figure}[H]
	\centering
	\includegraphics[width=1.\linewidth]{"tex/6-ac227-images/BNL/RnPoDt_TimeBin23_S2"}
	\caption{}
	\label{fig:rnpodttimebin23s2}
\end{figure}

\begin{figure}[H]
	\centering
\begin{subfigure}{0.5\linewidth}
	\centering
	\includegraphics[width=1.\linewidth]{"tex/6-ac227-images/BNL/RnPoEn_TimeBin23_S2"}
	\caption{}
\end{subfigure}%
\begin{subfigure}{0.5\linewidth}
	\centering
	\includegraphics[width=1.\linewidth]{"tex/6-ac227-images/BNL/RnPoPSD_TimeBin23_S2"}
	\caption{}
\end{subfigure}
	\caption{}
	\label{fig:rnpoenpsd}
\end{figure}

\subsection{Results}

\begin{figure}[H]
	\centering
	\includegraphics[width=1\linewidth]{tex/6-ac227-images/BNL/RateVsTime_AllSamples}
	\caption{}
	\label{fig:ratevstimeallsamples}
\end{figure}

\begin{figure}[H]
	\centering
	\includegraphics[width=1\linewidth]{tex/6-ac227-images/BNL/RelRateVsTime_AllSamples}
	\caption{}
	\label{fig:relratevstimeallsamples}
\end{figure}

\begin{figure}[H]
	\centering
	\includegraphics[width=1.\linewidth]{"tex/6-ac227-images/BNL/RnPoEn_FirstAndLast_S8"}
	\caption{}
	\label{fig:rnpoenfirstandlasts8}
\end{figure}

\begin{figure}[h]
	\centering
	\includegraphics[width=1\linewidth]{tex/6-ac227-images/BNL/RnPoEn_FirstAndLast_S6}
	\caption{}
	\label{fig:rnpoenfirstandlasts6}
\end{figure}



\begin{table}[H]
	\centering
\begin{tabular}{|c|c|c|}
	\hline 
	Sample & Constant & $\chi^2$/NDF \\ 
	\hline 
	3 -  & 0.997 $\pm$ 0.003 & 56.5/20 = 2.82 \\ 
	\hline 
	4 - & 1.005 $\pm$ 0.003 & 44.1/20 = 2.21 \\ 
	\hline 
	5 - & 1.003 $\pm$ 0.003 & 80.6/20 = 4.03 \\ 
	\hline 
	6 - & 0.985 $\pm$ 0.003 & 70.9/20 = 3.55 \\ 
	\hline 
	7 - & 1.001 $\pm$ 0.003 & 38.3/21 = 1.82 \\ 
	\hline 
	8 -  & 0.960 $\pm$ 0.002 & 136.3/21 = 6.49 \\ 
	\hline 
\end{tabular} 
\end{table}

\begin{table}[H]
	\centering
\begin{tabular}{|c|c|c|c|}
	\hline 
	Sample & Constant & Slope [ratio/yr] & $\chi^2/NDF$ \\ 
	\hline 
	3 -  & 1.5 $\pm$ 0.8 & -0.01 $\pm$ 0.02 & 56.2/19 = 2.96 \\ 
	\hline 
	4 -  & 1.3 $\pm$ 0.8 & -0.005 $\pm$ 0.018 & 44.0/19 = 2.32 \\ 
	\hline 
	5 -  & 4.4 $\pm$ 0.8 & -0.07 $\pm$ 0.02 & 64.3/19 = 3.38 \\ 
	\hline 
	6 -  & 2.9 $\pm$ 0.8 & -0.04 $\pm$ 0.02 & 65.2/19 = 3.43 \\ 
	\hline 
	7 -  & 2.3 $\pm$ 0.8 & -0.03 $\pm$ 0.02 & 35.7/20 = 1.79\\ 
	\hline 
	8 -  & 7.7 $\pm$ 0.7 & -0.14 $\pm$ 0.02 & 48.9/20 = 2.45 \\ 
	\hline 
\end{tabular} 
\end{table}


\section{Event Selection in the PROSPECT AD}

\section{Detector Stability Results}

\section{Volume Variation Results}

